% Options for packages loaded elsewhere
\PassOptionsToPackage{unicode}{hyperref}
\PassOptionsToPackage{hyphens}{url}
%
\documentclass[
]{article}
\usepackage{lmodern}
\usepackage{amssymb,amsmath}
\usepackage{ifxetex,ifluatex}
\ifnum 0\ifxetex 1\fi\ifluatex 1\fi=0 % if pdftex
  \usepackage[T1]{fontenc}
  \usepackage[utf8]{inputenc}
  \usepackage{textcomp} % provide euro and other symbols
\else % if luatex or xetex
  \usepackage{unicode-math}
  \defaultfontfeatures{Scale=MatchLowercase}
  \defaultfontfeatures[\rmfamily]{Ligatures=TeX,Scale=1}
\fi
% Use upquote if available, for straight quotes in verbatim environments
\IfFileExists{upquote.sty}{\usepackage{upquote}}{}
\IfFileExists{microtype.sty}{% use microtype if available
  \usepackage[]{microtype}
  \UseMicrotypeSet[protrusion]{basicmath} % disable protrusion for tt fonts
}{}
\makeatletter
\@ifundefined{KOMAClassName}{% if non-KOMA class
  \IfFileExists{parskip.sty}{%
    \usepackage{parskip}
  }{% else
    \setlength{\parindent}{0pt}
    \setlength{\parskip}{6pt plus 2pt minus 1pt}}
}{% if KOMA class
  \KOMAoptions{parskip=half}}
\makeatother
\usepackage{xcolor}
\IfFileExists{xurl.sty}{\usepackage{xurl}}{} % add URL line breaks if available
\IfFileExists{bookmark.sty}{\usepackage{bookmark}}{\usepackage{hyperref}}
\hypersetup{
  pdftitle={Machine Learning Engineer Nanodegree},
  pdfauthor={Florian Sckade},
  hidelinks,
  pdfcreator={LaTeX via pandoc}}
\urlstyle{same} % disable monospaced font for URLs
\usepackage[margin=1in]{geometry}
\usepackage{color}
\usepackage{fancyvrb}
\newcommand{\VerbBar}{|}
\newcommand{\VERB}{\Verb[commandchars=\\\{\}]}
\DefineVerbatimEnvironment{Highlighting}{Verbatim}{commandchars=\\\{\}}
% Add ',fontsize=\small' for more characters per line
\usepackage{framed}
\definecolor{shadecolor}{RGB}{248,248,248}
\newenvironment{Shaded}{\begin{snugshade}}{\end{snugshade}}
\newcommand{\AlertTok}[1]{\textcolor[rgb]{0.94,0.16,0.16}{#1}}
\newcommand{\AnnotationTok}[1]{\textcolor[rgb]{0.56,0.35,0.01}{\textbf{\textit{#1}}}}
\newcommand{\AttributeTok}[1]{\textcolor[rgb]{0.77,0.63,0.00}{#1}}
\newcommand{\BaseNTok}[1]{\textcolor[rgb]{0.00,0.00,0.81}{#1}}
\newcommand{\BuiltInTok}[1]{#1}
\newcommand{\CharTok}[1]{\textcolor[rgb]{0.31,0.60,0.02}{#1}}
\newcommand{\CommentTok}[1]{\textcolor[rgb]{0.56,0.35,0.01}{\textit{#1}}}
\newcommand{\CommentVarTok}[1]{\textcolor[rgb]{0.56,0.35,0.01}{\textbf{\textit{#1}}}}
\newcommand{\ConstantTok}[1]{\textcolor[rgb]{0.00,0.00,0.00}{#1}}
\newcommand{\ControlFlowTok}[1]{\textcolor[rgb]{0.13,0.29,0.53}{\textbf{#1}}}
\newcommand{\DataTypeTok}[1]{\textcolor[rgb]{0.13,0.29,0.53}{#1}}
\newcommand{\DecValTok}[1]{\textcolor[rgb]{0.00,0.00,0.81}{#1}}
\newcommand{\DocumentationTok}[1]{\textcolor[rgb]{0.56,0.35,0.01}{\textbf{\textit{#1}}}}
\newcommand{\ErrorTok}[1]{\textcolor[rgb]{0.64,0.00,0.00}{\textbf{#1}}}
\newcommand{\ExtensionTok}[1]{#1}
\newcommand{\FloatTok}[1]{\textcolor[rgb]{0.00,0.00,0.81}{#1}}
\newcommand{\FunctionTok}[1]{\textcolor[rgb]{0.00,0.00,0.00}{#1}}
\newcommand{\ImportTok}[1]{#1}
\newcommand{\InformationTok}[1]{\textcolor[rgb]{0.56,0.35,0.01}{\textbf{\textit{#1}}}}
\newcommand{\KeywordTok}[1]{\textcolor[rgb]{0.13,0.29,0.53}{\textbf{#1}}}
\newcommand{\NormalTok}[1]{#1}
\newcommand{\OperatorTok}[1]{\textcolor[rgb]{0.81,0.36,0.00}{\textbf{#1}}}
\newcommand{\OtherTok}[1]{\textcolor[rgb]{0.56,0.35,0.01}{#1}}
\newcommand{\PreprocessorTok}[1]{\textcolor[rgb]{0.56,0.35,0.01}{\textit{#1}}}
\newcommand{\RegionMarkerTok}[1]{#1}
\newcommand{\SpecialCharTok}[1]{\textcolor[rgb]{0.00,0.00,0.00}{#1}}
\newcommand{\SpecialStringTok}[1]{\textcolor[rgb]{0.31,0.60,0.02}{#1}}
\newcommand{\StringTok}[1]{\textcolor[rgb]{0.31,0.60,0.02}{#1}}
\newcommand{\VariableTok}[1]{\textcolor[rgb]{0.00,0.00,0.00}{#1}}
\newcommand{\VerbatimStringTok}[1]{\textcolor[rgb]{0.31,0.60,0.02}{#1}}
\newcommand{\WarningTok}[1]{\textcolor[rgb]{0.56,0.35,0.01}{\textbf{\textit{#1}}}}
\usepackage{graphicx,grffile}
\makeatletter
\def\maxwidth{\ifdim\Gin@nat@width>\linewidth\linewidth\else\Gin@nat@width\fi}
\def\maxheight{\ifdim\Gin@nat@height>\textheight\textheight\else\Gin@nat@height\fi}
\makeatother
% Scale images if necessary, so that they will not overflow the page
% margins by default, and it is still possible to overwrite the defaults
% using explicit options in \includegraphics[width, height, ...]{}
\setkeys{Gin}{width=\maxwidth,height=\maxheight,keepaspectratio}
% Set default figure placement to htbp
\makeatletter
\def\fps@figure{htbp}
\makeatother
\setlength{\emergencystretch}{3em} % prevent overfull lines
\providecommand{\tightlist}{%
  \setlength{\itemsep}{0pt}\setlength{\parskip}{0pt}}
\setcounter{secnumdepth}{-\maxdimen} % remove section numbering
\usepackage{booktabs}
\usepackage{longtable}
\usepackage{array}
\usepackage{multirow}
\usepackage{wrapfig}
\usepackage{float}
\usepackage{colortbl}
\usepackage{pdflscape}
\usepackage{tabu}
\usepackage{threeparttable}
\usepackage{threeparttablex}
\usepackage[normalem]{ulem}
\usepackage{makecell}
\usepackage{xcolor}

\title{Machine Learning Engineer Nanodegree}
\author{Florian Sckade}
\date{April 17th, 2020}

\begin{document}
\maketitle

\hypertarget{capstone-project-prediction-of-hotel-booking-cancellations}{%
\section{Capstone Project: Prediction of Hotel Booking
Cancellations}\label{capstone-project-prediction-of-hotel-booking-cancellations}}

\hypertarget{i.-definition}{%
\subsection{I. Definition}\label{i.-definition}}

\hypertarget{project-overview}{%
\subsubsection{Project Overview}\label{project-overview}}

This project seeks to adress a major point of planning efforts in
hospitality management: the prediction of booking cancellations. Within
the project, historical data of cancelled and checked-in hotel bookings
is explored for interesting relationships and utilized in machine
learning models to predict whether a customer will cancel their booking.

The dataset used for this project will be the Hotel booking demand
dataset (Mostipak 2020), hosted on kaggle.com, originally published in
Antonio, Almeida, and Nunes (2019). It contains booking information from
two different portugese hotels, one based in a city, the other being a
resort hotel. The focus will lie on the target variable - which
describes whether a booking was cancelled before the customer arrived or
not. Additionally, the dataset includes information about the booking
and its associated customers - like the number of adults and children,
over which travel agent the booking was made, the date of the booking or
the average daily rate for the customer.

A hotel needs to take future cancellations into account when they allow
a customer to book a room. Due to this, a hotel tends to be overbooked
by default, building upon the assumption that some customers will not
actually arrive. If a hotel predicts cancellations inaccurately, by
overestimating the actual number of canceled bookings, customers would
be need to be turned away upon their arrival. Similarily, if the
predicted number is undererstimating the actual number of cancellations,
the hotel could operate on too little capacity and lose money. As such,
the prediction of cancellations is a forecasting problem with high
business impact.

\hypertarget{problem-statement}{%
\subsubsection{Problem Statement}\label{problem-statement}}

Predicting churn is a very important part of every subscription based
business, which is often addressed by modern machine learning solutions.
Prominent examples are often found in telecommunications (Huang,
Kechadi, and Buckley 2012). The general problem formulation can be
transferred to other domains, for example hospitality management. Here,
churn translates to cancellations of bookings. Since this translates
very directly to a loss of revenue, the accurate prediction of
cancellations is of very high importance for such businesses.

For hospitality management, the accurate prediction of a cancellations
prior to the anticipated check-in date is very important. In this
context, the problem to be solved is the prediction of a cancellation
probability \(P(Y_i \vert X_i)\) of a customer \(i\), given a set of
features \(X_i\). Thus, the problem is a two-class classification
problem, allowing the performance of solutions to be quantitatively
evaluated by evaluation metrics like accuracy, precision or recall.
These solutions are two-class classification models. An implementation
should be able to accurately and repeatedly classify customers based on
the given features. Along with accurate predictions, it can be of major
importance to identify possible drivers of cancellations, so that
possible opportunities for actions can be derived from a machine
learning solution. As an example, a prediction model could generate next
best offers for customers at the point of booking, possibly reducing the
probability of a later cancellation.

\hypertarget{metrics}{%
\subsubsection{Metrics}\label{metrics}}

The project's solution will be evaluated against the
area-under-the-ROC-curve (AUC) score, as well as the accuracy. The
ROC-curve is the model's recall against the false positive rate, which
is equal to \((1 - Specificity)\) at various threshold settings. The AUC
is calculated for the ROC-curve to give a total score for the model.
Additionally, different classification measures and their importance and
implication in the context of the underlying business case will be
discussed.

Following Fawcett (2006), the accuracy is defined as
\[\text{Accuracy} = \frac{\text{True Positives } + \text{ True Negatives}}{\text{Positive } + \text{ Negatives}}\],
where the nominators are equivalent to correctly classifying positive
and negative observations respectively, while the denominator's
variables are the numbers of real positive and negative observations in
the data.\\
Recall and specificity, which are needed for calculating the ROC-AUC
score, are defined as
\[\text{Recall} = \frac{\text{True Positives}}{\text{True Positives } + \text{ False Negatives}},\]
and\\
\[\text{Specificity} = \frac{\text{True Negatives}}{\text{True Negatives + False Positives}}\]\\
respectively. Again, \emph{True Positives} denote the number of
correctly classified positive cases, i.e.~correctly predicted booking
cancellations, while \emph{True Negatives} denote the number of
correctly classified negative cases, i.e.~customers who were predicted
not to cancel their bookings and did not cancel. \emph{False Negatives}
and \emph{False Positives} are wrongly classified negative and positive
cases respectively.

For this problem, it is also worth considering precision, defined as
\[\text{Precision} = \frac{\text{True Positives}}{\text{True Positives + False Positvies}}\].
This is important, because it is valid to assume cancellations to occur
much less frequently in booking data than actual check-ins. Where
accuracy allows for measurement of the model's ability to seperate the
two classes, precision tells us how well the model is at predicting the
less-often occuring class.

\hypertarget{ii.-analysis}{%
\subsection{II. Analysis}\label{ii.-analysis}}

\emph{(approx. 2-4 pages)}

\hypertarget{data-exploration}{%
\subsubsection{Data Exploration}\label{data-exploration}}

In this section, you will be expected to analyze the data you are using
for the problem. This data can either be in the form of a dataset (or
datasets), input data (or input files), or even an environment. The type
of data should be thoroughly described and, if possible, have basic
statistics and information presented (such as discussion of input
features or defining characteristics about the input or environment).
Any abnormalities or interesting qualities about the data that may need
to be addressed have been identified (such as features that need to be
transformed or the possibility of outliers). Questions to ask yourself
when writing this section: - \emph{If a dataset is present for this
problem, have you thoroughly discussed certain features about the
dataset? Has a data sample been provided to the reader?} - \emph{If a
dataset is present for this problem, are statistics about the dataset
calculated and reported? Have any relevant results from this calculation
been discussed?} - \emph{If a dataset is \textbf{not} present for this
problem, has discussion been made about the input space or input data
for your problem?} - \emph{Are there any abnormalities or
characteristics about the input space or dataset that need to be
addressed? (categorical variables, missing values, outliers, etc.)}

\begin{Shaded}
\begin{Highlighting}[]
\NormalTok{df <-}\StringTok{ }\KeywordTok{read.csv}\NormalTok{(}\StringTok{"VarDesc.csv"}\NormalTok{)}
\NormalTok{kableExtra}\OperatorTok{::}\KeywordTok{kable}\NormalTok{(df, }\DataTypeTok{caption =} \StringTok{"Overview of the variables present in the dataset"}\NormalTok{)}
\end{Highlighting}
\end{Shaded}

\begin{table}

\caption{\label{tab:unnamed-chunk-1}Overview of the variables present in the dataset}
\centering
\begin{tabular}[t]{l|l|l}
\hline
Variable & Type & Description\\
\hline
hotel & categorical & Type of hotel, H1 = Resort Hotel or H2 = City Hotel\\
\hline
is\_canceled & integer & Value indicating if the booking was canceled (1) or not(0)\\
\hline
lead\_time & integer & Number of days between booking and arrival date\\
\hline
arrival\_date\_year & integer & Year of arrival date\\
\hline
arrival\_date\_month & categorical & Month of arrival date\\
\hline
arrival\_date\_week\_number & integer & Week number of year for arrival date\\
\hline
arrival\_date\_day\_of\_month & integer & Day of arrival date\\
\hline
stays\_in\_weekend\_nights & integer & Number of weekend nights the guest stayed or booked\\
\hline
stays\_in\_week\_nights & integer & Number of week nights the guest stayed or booked\\
\hline
adults & integer & Number of adults\\
\hline
children & integer & Number of children\\
\hline
babies & integer & Number of babies\\
\hline
meal & categorical & Type of meal booked\\
\hline
country & categorical & Country of origin\\
\hline
market\_segment & categorical & Market segment designation\\
\hline
distribution\_channel & categorical & Booking distribution channel\\
\hline
is\_repeated\_guest & integer & Value indicating if the booking name was from a repeated guest (1) or not (0)\\
\hline
previous\_cancellations & integer & Number of previous bookings that were cancelled\\
\hline
previous\_bookings\_not\_canceled & integer & Number of previous bookings that were not cancelled\\
\hline
reserved\_room\_type & categorical & Code of room type reserved\\
\hline
assigned\_room\_type & categorical & Code of room type assigned due to hotel operation reasons\\
\hline
booking\_changes & integer & Number of changes made to the booking\\
\hline
deposit\_type & categorical & Indication on if the customer made a deposit to guarantee the booking\\
\hline
agent & categorical & ID of the travel agency that made the booking\\
\hline
company & categorical & ID of the company that made the booking\\
\hline
days\_in\_wating\_list & integer & Number of days the booking was in the waiting list\\
\hline
customer\_type & categorical & Type of booking, one of: Contract, Group, Transient, Transient-party\\
\hline
adr & float & Average daily rate\\
\hline
required\_car\_parking\_spaces & integer & Number of car parking spaces required by the customer\\
\hline
total\_of\_special\_requests & integer & Number of special requests made by the customer\\
\hline
reservation\_status & categorical & Reservation last status, one of: Canceled, Check-Out, No-Show\\
\hline
reservation\_status\_date & categorical & Date at which the last status was set\\
\hline
\end{tabular}
\end{table}

\hypertarget{exploratory-visualization}{%
\subsubsection{Exploratory
Visualization}\label{exploratory-visualization}}

In this section, you will need to provide some form of visualization
that summarizes or extracts a relevant characteristic or feature about
the data. The visualization should adequately support the data being
used. Discuss why this visualization was chosen and how it is relevant.
Questions to ask yourself when writing this section: - \emph{Have you
visualized a relevant characteristic or feature about the dataset or
input data?} - \emph{Is the visualization thoroughly analyzed and
discussed?} - \emph{If a plot is provided, are the axes, title, and
datum clearly defined?}

\hypertarget{algorithms-and-techniques}{%
\subsubsection{Algorithms and
Techniques}\label{algorithms-and-techniques}}

In this section, you will need to discuss the algorithms and techniques
you intend to use for solving the problem. You should justify the use of
each one based on the characteristics of the problem and the problem
domain. Questions to ask yourself when writing this section: - \emph{Are
the algorithms you will use, including any default variables/parameters
in the project clearly defined?} - \emph{Are the techniques to be used
thoroughly discussed and justified?} - \emph{Is it made clear how the
input data or datasets will be handled by the algorithms and techniques
chosen?}

\hypertarget{benchmark}{%
\subsubsection{Benchmark}\label{benchmark}}

In this section, you will need to provide a clearly defined benchmark
result or threshold for comparing across performances obtained by your
solution. The reasoning behind the benchmark (in the case where it is
not an established result) should be discussed. Questions to ask
yourself when writing this section: - \emph{Has some result or value
been provided that acts as a benchmark for measuring performance?} -
\emph{Is it clear how this result or value was obtained (whether by data
or by hypothesis)?}

\hypertarget{iii.-methodology}{%
\subsection{III. Methodology}\label{iii.-methodology}}

\emph{(approx. 3-5 pages)}

\hypertarget{data-preprocessing}{%
\subsubsection{Data Preprocessing}\label{data-preprocessing}}

In this section, all of your preprocessing steps will need to be clearly
documented, if any were necessary. From the previous section, any of the
abnormalities or characteristics that you identified about the dataset
will be addressed and corrected here. Questions to ask yourself when
writing this section: - \emph{If the algorithms chosen require
preprocessing steps like feature selection or feature transformations,
have they been properly documented?} - \emph{Based on the \textbf{Data
Exploration} section, if there were abnormalities or characteristics
that needed to be addressed, have they been properly corrected?} -
\emph{If no preprocessing is needed, has it been made clear why?}

\hypertarget{implementation}{%
\subsubsection{Implementation}\label{implementation}}

In this section, the process for which metrics, algorithms, and
techniques that you implemented for the given data will need to be
clearly documented. It should be abundantly clear how the implementation
was carried out, and discussion should be made regarding any
complications that occurred during this process. Questions to ask
yourself when writing this section: - \emph{Is it made clear how the
algorithms and techniques were implemented with the given datasets or
input data?} - \emph{Were there any complications with the original
metrics or techniques that required changing prior to acquiring a
solution?} - \emph{Was there any part of the coding process (e.g.,
writing complicated functions) that should be documented?}

\hypertarget{refinement}{%
\subsubsection{Refinement}\label{refinement}}

In this section, you will need to discuss the process of improvement you
made upon the algorithms and techniques you used in your implementation.
For example, adjusting parameters for certain models to acquire improved
solutions would fall under the refinement category. Your initial and
final solutions should be reported, as well as any significant
intermediate results as necessary. Questions to ask yourself when
writing this section: - \emph{Has an initial solution been found and
clearly reported?} - \emph{Is the process of improvement clearly
documented, such as what techniques were used?} - \emph{Are intermediate
and final solutions clearly reported as the process is improved?}

\hypertarget{iv.-results}{%
\subsection{IV. Results}\label{iv.-results}}

\emph{(approx. 2-3 pages)}

\hypertarget{model-evaluation-and-validation}{%
\subsubsection{Model Evaluation and
Validation}\label{model-evaluation-and-validation}}

In this section, the final model and any supporting qualities should be
evaluated in detail. It should be clear how the final model was derived
and why this model was chosen. In addition, some type of analysis should
be used to validate the robustness of this model and its solution, such
as manipulating the input data or environment to see how the model's
solution is affected (this is called sensitivity analysis). Questions to
ask yourself when writing this section: - \emph{Is the final model
reasonable and aligning with solution expectations? Are the final
parameters of the model appropriate?} - \emph{Has the final model been
tested with various inputs to evaluate whether the model generalizes
well to unseen data?} - \emph{Is the model robust enough for the
problem? Do small perturbations (changes) in training data or the input
space greatly affect the results?} - \emph{Can results found from the
model be trusted?}

\hypertarget{justification}{%
\subsubsection{Justification}\label{justification}}

In this section, your model's final solution and its results should be
compared to the benchmark you established earlier in the project using
some type of statistical analysis. You should also justify whether these
results and the solution are significant enough to have solved the
problem posed in the project. Questions to ask yourself when writing
this section: - \emph{Are the final results found stronger than the
benchmark result reported earlier?} - \emph{Have you thoroughly analyzed
and discussed the final solution?} - \emph{Is the final solution
significant enough to have solved the problem?}

\hypertarget{v.-conclusion}{%
\subsection{V. Conclusion}\label{v.-conclusion}}

\emph{(approx. 1-2 pages)}

\hypertarget{free-form-visualization}{%
\subsubsection{Free-Form Visualization}\label{free-form-visualization}}

In this section, you will need to provide some form of visualization
that emphasizes an important quality about the project. It is much more
free-form, but should reasonably support a significant result or
characteristic about the problem that you want to discuss. Questions to
ask yourself when writing this section: - \emph{Have you visualized a
relevant or important quality about the problem, dataset, input data, or
results?} - \emph{Is the visualization thoroughly analyzed and
discussed?} - \emph{If a plot is provided, are the axes, title, and
datum clearly defined?}

\hypertarget{reflection}{%
\subsubsection{Reflection}\label{reflection}}

In this section, you will summarize the entire end-to-end problem
solution and discuss one or two particular aspects of the project you
found interesting or difficult. You are expected to reflect on the
project as a whole to show that you have a firm understanding of the
entire process employed in your work. Questions to ask yourself when
writing this section: - \emph{Have you thoroughly summarized the entire
process you used for this project?} - \emph{Were there any interesting
aspects of the project?} - \emph{Were there any difficult aspects of the
project?} - \emph{Does the final model and solution fit your
expectations for the problem, and should it be used in a general setting
to solve these types of problems?}

\hypertarget{improvement}{%
\subsubsection{Improvement}\label{improvement}}

In this section, you will need to provide discussion as to how one
aspect of the implementation you designed could be improved. As an
example, consider ways your implementation can be made more general, and
what would need to be modified. You do not need to make this
improvement, but the potential solutions resulting from these changes
are considered and compared/contrasted to your current solution.
Questions to ask yourself when writing this section: - \emph{Are there
further improvements that could be made on the algorithms or techniques
you used in this project?} - \emph{Were there algorithms or techniques
you researched that you did not know how to implement, but would
consider using if you knew how?} - \emph{If you used your final solution
as the new benchmark, do you think an even better solution exists?}

\begin{center}\rule{0.5\linewidth}{0.5pt}\end{center}

\textbf{Before submitting, ask yourself. . .}

\begin{itemize}
\tightlist
\item
  Does the project report you've written follow a well-organized
  structure similar to that of the project template?
\item
  Is each section (particularly \textbf{Analysis} and
  \textbf{Methodology}) written in a clear, concise and specific
  fashion? Are there any ambiguous terms or phrases that need
  clarification?
\item
  Would the intended audience of your project be able to understand your
  analysis, methods, and results?
\item
  Have you properly proof-read your project report to assure there are
  minimal grammatical and spelling mistakes?
\item
  Are all the resources used for this project correctly cited and
  referenced?
\item
  Is the code that implements your solution easily readable and properly
  commented?
\item
  Does the code execute without error and produce results similar to
  those reported?
\end{itemize}

\hypertarget{refs}{}
\leavevmode\hypertarget{ref-antonio2019hotel}{}%
Antonio, Nuno, Ana de Almeida, and Luis Nunes. 2019. ``Hotel Booking
Demand Datasets.'' \emph{Data in Brief} 22: 41--49.

\leavevmode\hypertarget{ref-fawcett2006introduction}{}%
Fawcett, Tom. 2006. ``An Introduction to Roc Analysis.'' \emph{Pattern
Recognition Letters} 27 (8): 861--74.

\leavevmode\hypertarget{ref-huang2012customer}{}%
Huang, Bingquan, Mohand Tahar Kechadi, and Brian Buckley. 2012.
``Customer Churn Prediction in Telecommunications.'' \emph{Expert
Systems with Applications} 39 (1): 1414--25.

\leavevmode\hypertarget{ref-kaggle}{}%
Mostipak, Jesse. 2020. ``Hotel Booking Demand.'' 2020.
\url{https://www.kaggle.com/jessemostipak/hotel-booking-demand}.

\end{document}
